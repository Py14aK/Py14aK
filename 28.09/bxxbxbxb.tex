							Teichmuller spaces and a complex bound to 
						11d supergravity

In order for a bound to exist for an action with infinite corrections when pertubated there must  exixts a harmonic map from R^n to R^n-1. This is very similar to Strogatskies theorem. 

Another interesting non mathematical argument for this existance is looking at bosonic string theory( which would contain the term of any gravitational theory since the nambu go to action IS defined as such).

In the mathematical field of differential geometry, a smooth map from one Riemannian manifold to another Riemannian manifold is called harmonic if its coordinate representatives satisfy a certain nonlinear partial differential equation. This partial differential equation for a mapping also arises as the Euler-Lagrange equation of a functional generalizing the Dirichlet energy (which is often itself called "Dirichlet energy"). As such, the theory of harmonic maps encompasses both the theory of unit-speed geodesics in Riemannian geometry, and the theory of harmonic functions on open subsets of Euclidean space and on Riemannian manifolds. 

Seeing as how M theory is a theory of (p,b)branes who satisfy  dirichlet(or neuman conditions) a 1 dimensional brane that is simply the Euler-Lagrange equation seems likely. We can view this in the context of string field theory as a loop feynman diagram. Again the Polyakova action is essentialy a path integral.

https://physics.stackexchange.com/questions/268782/why-are-the-nambu-goto-action-and-polyakov-action-equivalent-at-quantum-level

For the purpose of this paper the Chern Simons corrections do not change our general method. We shall right the action off 11d gravity as $$Intergral(sqrt(r))d12 $$

$$figure(1)$$

The spectrum of string that end on D branes(which can be thought as filling The R sphere)is tachyon free.

We are only interested in the bosonic section and will add o^2 for corrections.

M theory is dimensionally reduced to 11d supergravity (becker 311), which also shows that it would need an infinite amound of correction to reach R^n


Informally, the Dirichlet energy of a mapping f from a Riemannian manifold M to a Riemannian manifold N can be thought of as the total amount that f "stretches" M in allocating each of its elements to a point of N. For instance, a rubber band which is stretched around a (smooth) stone can be mathematically formalized as a mapping from the points on the unstretched band to the surface of the stone. The unstretched band and stone are given Riemannian metrics as embedded submanifolds of three-dimensional Euclidean space; the Dirichlet energy of such a mapping is then a formalization of the notion of the total tension involved. Harmonicity of such a mapping means that, given any hypothetical way of physically deforming the given stretch, the tension (when considered as a function of time) has first derivative zero when the deformation begins.

The theory of harmonic maps was initiated in 1964 by James Eells and Joseph Sampson, who showed that in certain geometric contexts, arbitrary smooth maps could be deformed into harmonic maps.[1] Their work was the inspiration for Richard Hamilton's first work on the Ricci flow. Harmonic maps and the associated harmonic map heat flow, in and of themselves, are among the most widely studied topics in the field of geometric analysis. 


https://link.springer.com/article/10.1007/s10013-018-0298-7

Now we define the genius or number of loops for our action.

The Witten genus is a genus with coefficients in power series in one variable, playing the role of a universal elliptic genus. This arises (Witten 87) as the large volume limit of the partition function of the superstring (hence in the string worldsheet perturbation theory about constant worldsheet configurations). Specifically, for the type II superstring this reproduces the universal elliptic genus as previously introduced by Serge Ochanine, while for the heterotic string it yields what is now called the Witten genus proper. Concretely, as Witten argued, this is a formal power series in string oscillation modes of the A-hat genus of the symmetric tensor powers of the tangent bundle that these modes take values in.

In (Witten 86) it is suggested, by regarding the superstring sigma-model as quantum mechanics on the smooth loop space of its target space, that the Witten genus may be thought of as the large volume limit of an S 1S^1-equivariant A-hat genus on smooth loop space, hence the index of the Dirac-Ramond operator in that limit. (Ever since this suggestion people have tried to make precise the concept of Dirac operator on a smooth loop space (e.g. Alvarez-Killingback-Mangano-Windey 87). But notice that, by the above, only the formal loop space and the Dirac-Ramond operator really appears in the definition of the Witten genus.)

A priori the coefficients of the Witten genus as a genus on oriented manifolds are formal power series over the rational numbers


Define string scattering amplitude via Feynman diagram (witten,green book 2)

Insert Renormalizibilty proof here

If the action needs is non-perturbatively renormalizable then the genus of the low efective theory isnt some crazy order of 24 but 2.(https://arxiv.org/pdf/1705.06777.pdf)

insert definition of harmonic maps.

We build a type of 11,1 Field thoery that is UV complete and time ordered(dyson series)=> which we will show gives us every single ADS(Weyl_Peterson metric with E8xE8 and all discovered lie gropu that are connected + a method for lattice fixing.) discovery from the past 20 years and predicts a Higgs bundle, explaining how electrons, the higgs mechanicsm and gravity are connected.


Define teichmuller covering such that the induced metric between mapings is R^6n-6 We use Tau to denote TM spaces 
c
Insert reference Teichmuller (Teichmüller Theory in Riemannian Geometry by Anthony J. Tromba )

Tau metric is isomoprhic to the 6d ADSxCFT metric =>which means its bounded by E6.

All Teichmuller spaces are defined as hyperbolic Insert reference (Teichmüller Theory in Riemannian Geometry by Anthony J. Tromba )

A perfect candidate for a DS/cft of evolutionary kind.

We shall show that the although hyperbolic surfaces allow of many metric they all are equivelent.The only important metrivs are WP metric and and th Einstein_kehler metric( another prediction or consistancy)

Define the S11 gravity as nablaHR^11 or the low energy action which by (Witten,green book 2)is of genus 2 which means that H is sobolev.

Define Sobolev space aka is A hilbert space

П(W,N)=1/4(fx|v_t|^2-f/4(fh)V^2

Def tangent space f(omega),y

Nabla H is Harmonic(Which makes sensedue to S symmetry,heat map, Riccy Flow)

c

Theorem: If k<1(which becaus it's and ADS metric isomoprhic the Weyl Peterson Metric) The opposite map  Traingle F  exists = existance of Stabilizators.

It so happens that the stabilizators for a genus 2 cover are namely

->A,B,C,D classical lie groups and G2,F4,E7,E8, which suprise suprise is basically what 30 yearstyf superstring theory in a nice wrap, the theory is at least consistent.

The second important consequence of using TM theory on 11D supergravity is that we can and will prove that the action does not diverge.. Let's do it now.

Theorem Korever_schur lemma is the opposite it says that the action bounded by a TM cover is divergent....just like our non pertubativly renormalizable supergravity. It also proves that we are dealing with a time ordered function very similar to a Feynman path integral

Order(f)=lim epsi bounded integral(bx)f |nablaf|^2dv omega

Where we define Omega via U((omega 

C(U)sqrt(E(t))



Teichmüller's theorem asserts the existence and uniqueness of the extremal quasiconformal map between two compact Riemann surfaces of the same genus modulo an equivalence relation. The equivalence classes form the Teichmüller space T_p of compact Riemann surfaces of genus p. 

The existance of a harmonic map(becker chapter 8) is enough to prove the existance of a TM space(Teichmüller Theory in Riemannian Geometry by Anthony J. Tromba)

it is defined via a geodesic or integral

simf-sim(tauTino)

TIno is a tangential space to our tao space Teichmüller Theory in Riemannian Geometry by Anthony J. Tromba

Ext Tino(tau)=Bounded integral |eF|

Define Higgs bundle (V,Ф) aka Tino(tangent to TM space).

A Higgs bundle is a holomorphic vector bundle EE together with a 1-form Φ\Phi with values in the endomorphisms of (the fibers of) EE, such that Φ∧Φ=0\Phi \wedge \Phi = 0.

Higgs bundles play a central role in nonabelian Hodge theory.

Let ℰ\mathcal{E} be a sheaf of sections of a holomorphic vector bundle EE on complex manifold MM with structure sheaf 𝒪 M\mathcal{O}_M and module of Kähler differentials Ω M 1\Omega^1_M.

A Higgs field on ℰ\mathcal{E} is an 𝒪 M\mathcal{O}_M-linear map

Ф:Epsilon->Covering*exterior product_0m Epsiol

Define Higgs boundary (dirichlet) 
i/2pi F nabla + [Ф,Ф^v] = SI exterior product k 

Theorem (D-D-W)  
(For an unbounded sequence with harmonic map of the p-type ⊆ )

https://en.wikipedia.org/wiki/Bolzano%E2%80%93Weierstrass_theorem 

Next we show there is a (present) metric, and it is the Einstein-Kahler metric.

W(K) = V(K) - 1/4 bounded integral partial derivative (k) HdA + 1/2L_(lamda)    

theorem (WP)
(Teichmüller Theory in Riemannian Geometry by Anthony J. Tromba)
=> W-P is unique 

ε^ + ε^- = id 

 1/4 bounded integral partial derivative (k) HdA 
 => Chern theory of dV
 => g=2  
 
 Friedmen geodesic

 Teichmuller space T(S)
 Existence of Heisenberg uncertainty principle
 This follows from g>2, which gives Sobolev space
 
 2.1.4 The Teichm ̈uller space.We now come to the definition of Teichm ̈ullerspace. LetSbe a closed Riemann surface of genusp≥2. Consider triples(S,f,R), whereRis a Riemann surface andf:S→Ris an orientationpreserving diffeomorphism. Triples (S,f1,R1) and (S,f2,R2) are said to beequivalent iff2◦f−11:R1→R2is homotopic to a biholomorphism. The setof all equivalence classes [S,f,R] of triples (S,f,R) is denotedT(S) and iscalled theTeichm ̈uller space based onS. The definition ofT(S) turns out tobe independent of the complex structure onS(see Theorem 2.7 below). Sinceany homeomorphism (in particular quasiconformal ones) is homotopic to a dif-feomorphism, one obtains the same space if one considers pairs (S,f,R) wherefis quasiconformal. This is a point of subtlety when dealing with Riemannsurfaces with punctures.
 
 Harmonic maps are also 	Biholomorphic, once to a teichmuler space and once to a conformal space
 
 The M string action will be defined as R^11XS^1 as stated above E(f). Note the slight diffrence between a dimensional reduction and KK compatification.
 
 Prove that theM theory action given by TM space is well convergent and exactly findable.(comes from the epsilon+ term)
 
 T*(S) metric_hyp=Methyp(S)/Diff_0(s)

 Theorem: Assume M,N are compact R manifolds and N has non positive sectional curvature Given a smooth map f: M->N, then the solution are lie between {0 and Inf).
 The key here being  the parabolic bochner theorem between smooth manifolds.
 
 Suppose f(x,t) is a solution for 0<t<T, let e(f)(x,t) denote the energy density of the 
 
 We get the following identity
 
 Partial derivative e(f0)/t trianglee(t)=|nabladf|^ +<df Ric^M(u_a),df(ua)> where Ric	^M Ricci curvature of (M,g).In particular	
 
 
 Theorem:For G>2 the teichmuller metric there exists a unique weyl_Peterson metric.
 
 This is a very powerfull theorem combined with the proper epsilon+_ maps because it demands a time ordered series.
 
 Being a Stein manifold is equivalent to being a (complex) strongly pseudoconvex manifold. The latter means that it has a strongly pseudoconvex (or plurisubharmonic)
 
 A major problem in finding the effective string action has been bizare number of choices for compactificatiob,,by using Tm compactification we escape this dilema.
 
the induced metric is 6g-6+2n

 
 F,F2 -+>e-t  F1(agen),F2) 
 	